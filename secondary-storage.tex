% THIS IS SIGPROC-SP.TEX - VERSION 3.1
% WORKS WITH V3.2SP OF ACM_PROC_ARTICLE-SP.CLS
% APRIL 2009
%
% It is an example file showing how to use the 'acm_proc_article-sp.cls' V3.2SP
% LaTeX2e document class file for Conference Proceedings submissions.
% ----------------------------------------------------------------------------------------------------------------
% This .tex file (and associated .cls V3.2SP) *DOES NOT* produce:
%       1) The Permission Statement
%       2) The Conference (location) Info information
%       3) The Copyright Line with ACM data
%       4) Page numbering
% ---------------------------------------------------------------------------------------------------------------
% It is an example which *does* use the .bib file (from which the .bbl file
% is produced).
% REMEMBER HOWEVER: After having produced the .bbl file,
% and prior to final submission,
% you need to 'insert'  your .bbl file into your source .tex file so as to provide
% ONE 'self-contained' source file.
%
% Questions regarding SIGS should be sent to
% Adrienne Griscti ---> griscti@acm.org
%
% Questions/suggestions regarding the guidelines, .tex and .cls files, etc. to
% Gerald Murray ---> murray@hq.acm.org
%
% For tracking purposes - this is V3.1SP - APRIL 2009

\documentclass{acm_proc_article-sp}
\usepackage[hyphens]{url}

\usepackage{eurosym}
%\usepackage[breaklinks,colorlinks=true,linkcolor=black,urlcolor=black,citecolor=black]{hyperref}

\makeatletter
\def\@copyrightspace{\relax}
\makeatother

\begin{document}

\title{Computer Secondary Storage}
\subtitle{Seminar: Use of Modern Hardware in Big Data Processing}
%
% You need the command \numberofauthors to handle the 'placement
% and alignment' of the authors beneath the title.
%
% For aesthetic reasons, we recommend 'three authors at a time'
% i.e. three 'name/affiliation blocks' be placed beneath the title.
%
% NOTE: You are NOT restricted in how many 'rows' of
% "name/affiliations" may appear. We just ask that you restrict
% the number of 'columns' to three.
%
% Because of the available 'opening page real-estate'
% we ask you to refrain from putting more than six authors
% (two rows with three columns) beneath the article title.
% More than six makes the first-page appear very cluttered indeed.
%
% Use the \alignauthor commands to handle the names
% and affiliations for an 'aesthetic maximum' of six authors.
% Add names, affiliations, addresses for
% the seventh etc. author(s) as the argument for the
% \additionalauthors command.
% These 'additional authors' will be output/set for you
% without further effort on your part as the last section in
% the body of your article BEFORE References or any Appendices.

\numberofauthors{1} %  in this sample file, there are a *total*
% of EIGHT authors. SIX appear on the 'first-page' (for formatting
% reasons) and the remaining two appear in the \additionalauthors section.
%
\author{
% You can go ahead and credit any number of authors here,
% e.g. one 'row of three' or two rows (consisting of one row of three
% and a second row of one, two or three).
%
% The command \alignauthor (no curly braces needed) should
% precede each author name, affiliation/snail-mail address and
% e-mail address. Additionally, tag each line of
% affiliation/address with \affaddr, and tag the
% e-mail address with \email.
%
% 1st. author
\alignauthor
Sascha K{\"o}nigsberg, Claus Strasburger\\
       \affaddr{Technische Universit{\"a}t M{\"u}nchen}\\
       \email{\{sascha.koenigsberg, c.strasburger\}@tum.de}
}

\date{20 May 2014}

\maketitle
\begin{abstract}
This report contains a summary of developments in computer secondary storage, particularly in big data server environments. We describe the history of \emph{Hard Drives} and \emph{Solid State Disks} and list tradeoffs and considerations of the currently available options. We present \emph{Phase Change Memory}, a possible alternative for the future of secondary storage, its history and current state of research. We provide a summary of the presented techniques and trends.%% fazit?
\end{abstract}

\section{Introduction}
When we talk about computer secondary storage, we should briefly explain what secondary storage is. As the name suggests there also exists primary storage in computers. This part of the memory hierarchy contains the CPU registers, CPU caches (cache levels one to three) and the main memory (\emph{DRAM}). It is directly accessible by the CPU but volatile on power loss. This means that all data are lost if the power supply is interrupted or the computer is shut down. Computer secondary storage is in the memory hierarchy one layer below the RAM. It includes flash memory (like SSD, USB flash drives), optical discs (CD-ROM, DVD), magnetic discs (HDD, Floppy) and magnetic tape (Compact Cassette). Secondary storage is also known as external memory and differs from primary storage in that it is not directly accessible by the CPU. The computer usually uses its input/output channels to access secondary storage. It is non-volatile --- it keeps data on power loss. In this report, we focus on hard disk drives and solid state drives.
\\
Hard disk drives use magnetic coated rotating disks (called \emph{platters}) to store data. The direction of magnetization represents binary data bits. The actuator head detects and modifies the magnetization. The actuator arm moves to the correct cylinder on the disc and waits for the platter to rotate until the data is below the actuator head. (TODO: MORE EXPLANATION?) 
\\
Solid state drives use two key components: The controller and the memory. The controller is an embedded processor and the connection between the host computer and the memory. It executes code on firmware-level and performs many functions necessary for SSD operation. These include Error-correction code (ECC), Wear-leveling, garbage collection and encryption.
\\
The controller is very important for the performance of a SSD. For example, if the controller provides multiple channels and can address all of them in parallel, multiple input/output operations can happen simultaneously on different memory chips, dramatically increasing the speed and response time. The memory is of course the location where data are saved on and usually consists of NAND flash , DRAM or a combination of both. NAND is cheaper and non-volatile, but it is slower than DRAM. Most SSDs are used as long-time storages requiring non-volatile memory, which means that most SSDs use NAND flash memory. These memory chips can be organized as Single-level cell (\emph{SLC}) or Multi-level cell (\emph{MLC}). SLC is ten times more persistent and allows three times faster sequential write with nearly the same sequential read performance, but it is 30\% more expensive. The usage of SLC or MLC chips is mostly determined by pricing: Cheap SSDs use MLC chips, while high performance enterprise SSDs use SLC.

\section{History of storage drives}

\subsection{Hard Disk Drives}
Hard disk drives are the successor to drum memory. The first HDD was the IBM 350. It had a capacity of 5 MB and was as big as a locker, because the actuator arm was moved electro-pneumatic. It was announced in 1956 and was not sold but rented for 650 dollar a month. Years later IBM started the Winchester project with the goal to develop a rotating memory with firmly mounted medium. The first HDD of the Winchester project was the IBM 3340 in 1979. The first Winchester drive in 5.25" form factor was the Seagate ST506 (in 1980). This name also became the name of the interface used by the ST506. The ST506-interface established as de-facto-standard.
\\
In 1986 SCSI was specified, one of the first official HDD protocols. Three years later, ATA, another HDD interface, was specified. In 1997 Giant Magnetoresistive Effect (\emph{GMR}) was used for the first time. This is a physical effect that appears within structures with alternately magnetic and non-magnetic layers and allowed to considerably increase the storage capacity.
\\
The first Serial ATA-HDD used Native Command Queueing (\emph{NCQ}) in 2004. NCQ is a technology to reorder the input/output operations and improve the performance. The first prototype of 2.5" Hybrid-HDD (\emph{H-HDD}) included NAND flash memory and was announced in 2005. The flash memory was used as cache to enhance the speed and the response time of the HDD. The 1 terabyte limit was reched by Hitachi in 2007, the 2 TB and 3 TB limit by Western Digital in 2009/2010. Important form factors of HDD are 5.25.", 3.5", 2.5" (used in laptops) and 1.8" (also used in laptops and some mobile media players, such as the Apple iPod). Important internal interfaces are ATA (IDE), ESDI, SCSI, S-ATA, Serial Attached SCSI (\emph{SAS}) and ST506 (important external interfaces: FireWire, USB and eSATA).
\\
While  there were multiple manufacturers on the HDD market for years, today only Seagate, Toshiba and Western Digital produce HDDs.

\subsection{Solid State Disks}
The origins of SSDs used magnetic core memory and card capacitor read-only store (CCROS). They were built in the 1950s. In the 1970s and 1980s SSD were used in supercomputers only. The first flash-based SSD were developed by MSystems in 1995, but it was so expensive, that it were only used for military reasons and other less price-sensitive sectors. In 1999 BiTMICRO announced the first SSD in 3.5" form factor. The first SSD for laptops in 2.5" and 1.8" form factors were developed by Samsung in 2006 and costz about 600 US-dollar. This is only an eighth of the price of previous SSDs. This wasn't the breakthrough for SSDs yet, but Samsung could gain 45% of market share. Only one year later Fusion-io announced the first PCIe-baes SSD which reached 100,000 IOPS (as comparisson: today HDD reaches up to 125 IOPS). The second generation of SSDs for consumers were the first generation that used MLC instead of SLC flash memory chips. As we have seen this decreased the price for SSDs, but the performancer was worse than SLC-based SSD's performance. In 2009 SSDs used improved MLC flash memory chips which had even better performance than some SLC chips. The term 'Enterprise flash drives (EFDs)' were introduced in 2008, but it became never a standard, so every SSD-manufacturer can name their SSDs as EFDs.

\section{Advantages and disadvantages of SSDs compared to HDDs}
As you can see in Table 1 SSDs have a very short access time, high data transfer rate and high IOPS. Because there are no moving parts, it is completely silent and very robust. This includes properties like shock resistance, temperature tolerance, dimensions, weight, etc. and leads to a very good usability in mobile computers. As we have seen when talking about SSD controllers, the speed can increase with a higher rate of parallelizability. The speed of SSDs can be so high, that even the newest and most advanced interfaces for SSDs and HDDs are too slow. So modern SSDs use interfaces like mSATA, PCI-Express and FibreChannel. Because of the small dimensions new form factory like mSATA, PCIe Mini Card and M.2 were developed too.
\\
But there are also some negative points about SSDs. SSDs are expensive; especially the price-per-GB compared to HDDs. Another problem is the limited time of write cycles for every NAND-cell. Later we will show you the disadvantages of Write Amplification and its complex countermeasures.
\\
%TODO: Insert Table 1 here

\section{SSD challenges}
Solid State Drives pose many difficult challenges to both their manufacturers and users. Despite the superior theoretical access time and read/write bandwidth of NAND flash, the inherent limitations of NAND flash memory make development and use of SSD difficult and completely different to regular hard disk drives.

\subsection{Hardware Limits}
NAND flash has two major limits: Block erasure and write cycles (\emph{Memory wear}). Other limitations include Bit Corruption and Read Disturb.

\subsubsection*{Block Erasure}
Because of the higher current which is necessary to release the trapped charge in a flash memory cell compared to charging a cell, this process is done in blocks of cells, called \emph{Erase Blocks}. Most current Solid State Disks use blocks of 256 pages or more \cite{codecapsule2014coding}.
This means that to write to a page which was previously written and subsequently marked as unused, either all pages on the same erase block have to be marked unused too, or currently in use pages have to be copied to another block. It is not until this has been done that the block can be erased. This process is called \emph{Garbage Collection}.

\subsubsection*{Limited Write Cycles}
When writing to a flash cell, charge is captured in it, similiar to a capacitor. %TODO similiar? the same? hmm.
However, when erasing it, not all charge is released. Some electrons get trapped in the gate, building up charge until the potential of an uncharged cell is indistinguishable from that of a charged cell. This phenomenon, known as \emph{Memory wear}, occurs earlier in multi level cells than in single level cells, since the voltage differences between levels are smaller and the cell is unusable as soon as the trapped charge approaches the difference between two levels.
Current SLC NAND flash cell lifetimes range from hundreds of thousands program/erase (\emph{P/E}) cycles to one million before becoming useless, while multi level or triple level cells have three thousand to ten thousand cycles.

\subsubsection*{Bit Corruption}
Raw flash memory is prone to random bit errors, because of charge leaking in and out of storage cells. Error correcting codes (\emph{ECC}), as well as error detecting codes (such as Cyclic Redundancy Check, \emph{CRC}) are used to find and possibly correct errors without silently corrupting data.

\subsubsection*{Read Disturb}
Another phenomenon in NAND flash is called \emph{Read Disturb}. When a cell is accessed many times without being erased, trapped charge can build up in nearby cells, causing them to flip their status if they were uncharged before.
This happens after about one million reads without erase in single level cells and after 100 thousand reads for multi level cells \cite{cooke2007inconvenient}.
\\
To prevent read disturb from corrupting data, SSD controllers keep count of how many times a page was read since the last erase cycle. When a predefined critical read limit is approached, the data is moved or re-programmed. Read disturb happening in spite of this prevention are handled by ECC.

\subsection{Controller Firmware Challenges}
SSD controllers have to do a lot of bookkeeping to prevent the underlying flash memory from corrupting data or deteriorating. Errors or oversights in controller firmware programming can lead to drastically reduced performance, data loss or silent data corruption.

\subsubsection*{Garbage Collection}
In order to write data to flash memory, free pages have to be available. SSD controllers use Over-Provisioning (\emph{OP}) to keep the number of unwritten pages high during burst writes, but if more data is to be written than free pages are available, or when invalidated pages need to be reclaimed, garbage collection is necessary. In this process, the controller copies still-valid pages out of blocks containing many invalid pages, only then the whole block can be erased. As the number of invalidated or free pages decreases --- to the operating system these are known as \emph{free} space --- more garbage collection and copying of pages in use has to be done in order to ensure available pages on write requests.

\subsubsection*{Wear Leveling}
Flash cells degrade when being written. When one section of the disk is written many times, e.g. a swap file or a simple counter, the page containing that section can wear out very quickly --- limiting the amount of overprovisioning or, in worse cases, reducing the effective size of the disk before it is at the end of its life. To prevent this from happening, the controller keeps count on page writes and re-maps heavily written pages to other pages regularly.

\subsubsection*{Write Amplification}
Summing up these techniques used by the controller, write accesses to the device can lead to more data being written than requested by the operating system. These include garbage collection, wear leveling, and most prominent, small and/or random writes.
Also, in the case of read disturb mitigation, even read accesses can cause writes to the disks.
In optimal situations, write amplification can be as low as a factor of $1.1\times$, while in worst-case conditions, the can go up to $10\times$ or $20\times$.
\\
Some manufacturers decrease this benchmark value by compressing and deduplicating data on-the-fly before writing it to disk, but this will only show effects for compressable data and only increase overhead for data not easily compressible.

\subsection{Software Challenges}
Even though SSD manufacturers go to great lengths to ensure optimal performance and data integrity, there are still considerations necessary for system architects and developers when building software or systems using SSDs. Blindly following current usage patterns established for rotating media will lead to bad performance, possibly worse than with hard drives, sudden performance drop or high device failure rate.

\subsubsection*{Random Writes}
Even though SSDs have no mechanical parts, sequential writes will still be faster than small writes. Applications have to be careful to try and write in as big blocks as possible, since the smallest written unit will always be the page size (currently 8 to 16 KiB). Small, random writes, such as those produced by a conventional Relational Database Management System (\emph{RDBMS}) like PostgreSQL or MySQL, will cause rapid allocation of many pages, which then need to be garbage collected. As soon as the SSD controller runs out of overprovisioned pages, garbage collection will be necessary and drastically slow down the device. The effective Write Amplification will be very high (depending on the write sizes and their ratio to the page and block sizes), also leading to faster memory degradation and therefore a high disk failure rate.

\subsubsection*{TRIM}
\emph{TRIM} is an ATA command used to mark unused pages. Conventional spinning hard drives have and need no concept of used or unused storage, since all they do is write and read. With SSDs, the controller needs to know what data is actually valid, otherwise it will do bookkeeping for data no longer referenced by the file system. Also, this storage space cannot be used by the controller to perform garbage collection, wear leveling or other optimization techniques and consequently slows down operation.
\\
To resolve this, the TRIM command has been introduced to the ATA specification to allow the operating system or file system to inform the drive when files are being unlinked. The SSD controller can then reclaim the now free space and use it at will.
\\
The lack of TRIM support was an issue when first introducing SSDs, it has however been implemented in all major operating systems since 2011. Still, some combinations of operating and file systems (for example NTFS-3g on Linux or Android before 4.3 \cite{androidtrim}) might lack TRIM support and therefore cannot take full advantage of SSD speeds.

\section{Parallel access mechanisms}
Block-oriented access over flash causes redundant writes. Redundant writes are bad for SSDs because they lower the overall performance of SSDs and decreases the reliability of flash memory. So mitigation is necessary when using SSDs. In this section we will show you some of the techniques to avoid redundant writes.

\subsection{B-Tree Layer for Flash Memory Storage Systems}


\cite{wu2007efficient}

\subsection{Fractal Prefetching B$^{+}$-Trees}

\cite{chen2002fractal}

\subsection{?} % microsoft case study?

\subsection{Flash as Cache Extension (\subsecit{FaCE})}

\cite{kang2012flash}

\section{Current Research}

\subsection{Phase Change Memory (\secit{PRAM})}

\subsection{Self-healing NAND Flash Memory}
\cite{wu2011exploiting}
\cite{chen2013dheating}

\section{Future Trends}
Current SSD prices are still high --- while, as of 2014, 1 TiB of enterprise SSD storage costs around \EUR{1,500} to \EUR{2,000} and a server with 1 TiB of main memory will cost from \EUR{10,000} to \EUR{15,000}, there is also a speed improvement of one order of magnitude in both access time and read/write speed. In a production environment, instead of extending the cache layer hierarchy by an SSD layer, the consideration of increasing main memory first is important, especially in cases where the working set could fit in main memory. It could be stated that maximizing main memory before investing in the necessary software to properly use SSD storage is preferable in most cases.
%TODO

%
% The following two commands are all you need in the
% initial runs of your .tex file to
% produce the bibliography for the citations in your paper.
\bibliographystyle{abbrv}
\bibliography{secondary-storage}  % sigproc.bib is the name of the Bibliography in this case
% You must have a proper ".bib" file
%  and remember to run:
% latex bibtex latex latex
% to resolve all references
%
% ACM needs 'a single self-contained file'!
%

\balancecolumns
\end{document}
